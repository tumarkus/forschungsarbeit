
\usepackage[utf8]{inputenc}   % Font Encoding, benoetigt fuer Umlaute
\usepackage[english]{babel}    % Spracheinstellung
\usepackage{caption}
\usepackage{graphicx}

\usepackage[T1]{fontenc}     % T1 Schrift Encoding
\usepackage{textcomp}         % Zusatzliche Symbole (Text Companion font extension)
\usepackage{lmodern}         % Latin Modern Schrift

\usepackage{amsmath,amssymb}  % mehr Mathe!
\usepackage{psfrag}           % Text in eps-Bildern ersetzen
\usepackage{url}              % Hyperlinks für urls
\usepackage{graphicx}
\usepackage{tabularx}
\usepackage{float}            % gleitende Objekte mit [H] wirklich HIER positionieren^
\usepackage{color}            % bring Farbe ins Leben =)
\usepackage{geometry}         % einfache Seiteneinrichtung
%\usepackage{subfigure}
%\usepackage{booktabs}
%\usepackage{framed}
\usepackage{fancyhdr}         % schicke Kopf- und Fußzeilen basteln
\usepackage{fancybox}	      % für runde Umrandungs-Boxen und solchen Crap
\usepackage{ifthen}	      % logische Abfragen
\usepackage{hyperref}         % interaktive hyperlinks für Querverweise
\usepackage{currfile}	      % get the current file name


%%%%%%% Flags for solutions: %%%%%%%%%%%%%%%%%%%%%%%%%%%%%%%%%%%%%%%%%%%
%\newboolean{Solution}
%\newboolean{longSol}



%%%%%% Seiteneinrichtung und Kopf/Fußzeilendefinition %%%%%%%%%%%%%%%%%%%
\geometry{a4paper,left=25mm,right=20mm,top=30mm,bottom=50mm}
% set indentation for paragraphs to zero:
\setlength{\parindent}{0pt}
% alle Kopf- und Fußzeilenfelder löschen:
\fancyhf{}
\setlength{\headwidth}{\textwidth}
\setlength{\headheight}{0pt}
% Linie in Kopfzeile ausblenden:
\renewcommand{\headrulewidth}{0.0pt}
\fancyfoot[R]{\sffamily \thepage}
% left footer only when detailed solutions are switched on:
%\ifthenelse{\boolean{longSol}}{
%  \fancyfoot[L]{\sffamily Aufgabe \theproblem}}{}
%% Linie Fußzeile:
%\renewcommand{\footrulewidth}{0.5pt}
%\renewcommand*\chapterheadstartvskip{\vspace{-\topskip}}
% Abstand zwischen Kapitelüberschrift und Seitenkopf verkleinern:
%\renewcommand*\chapterheadstartvskip{\vspace{-8ex}}

% Standardseitenstil (alle Seiten ohne Kapitelüberschrift) definieren:
%\fancyhead[L]{\iomHeader}

% plain-Seitenstil (für Seiten mit Kapitelüberschrift) überschreiben, so dass auch die erste Kapitelseite Kopf-und Fußzeile nach Maß bekommt:
%\fancypagestyle{plain}{\fancyhead[L]{\iomHeader}}

% Kopfzeile für IOM:
%\newcommand{\iomHeader}{
%  \sffamily
%  \begin{tabularx}{\textwidth}{lXr}
%  \hline
%  \,\includegraphics[height= 0.27cm]{imlogo} 
%  TU Dortmund -- Institut für Mechanik & & Prof. Dr.-Ing. Andreas Menzel\\   Mechanik für Maschinenbau & & Dipl.-Ing. Raphael Holtermann\\
%  \hline\\
%  \end{tabularx}
%}
%%%%%%%%%%%%%%%%%%%%%%%%%%%%%%%%%%%%%%%%%%%%%%%%%%%%%%%%%%%%%%%%%%%%%%%%%%

% diverses Zeug
\definecolor{greend}{rgb}{0.00,0.60,0.00} % dark grey
\definecolor{blued}{rgb}{0.00,0.00,0.60} % dark blue

\renewcommand{\vec}[1]{\ensuremath{\mbox{\boldmath $#1$}}}
\renewcommand{\d}{\operatorname{d}\negthinspace}
\newcommand{\smallsep}{\setlength{\itemsep}{-0.1\parsep}}
\newcommand{\bs}{\boldsymbol}

\newcommand{\eigenname}[1]{\textsc{#1}}
\newcommand{\begriff}[1]{\textbf{#1}}
\newcommand{\buchtitel}[1]{"\textit{#1}"}
\numberwithin{equation}{section}
\numberwithin{figure}{section}


%%%%% Befehlsumgebungen für Aufgaben: %%%%%%%%%%%%%%%%%%%%%%%%%%%%%%%%%%%%%
%
%% Zähler für Aufgaben:
%\newcounter{problem}
%% der bei jedem neuen Kapitel  0 gesetzt wird:
%\numberwithin{problem}{chapter}
%
%% Befehl für Aufgaben:
%\newcommand{\problem}[4][width=0.8 \textwidth]{
% \addtocounter{problem}{1}
% 
% %insert page break if longsolution is provided
% \ifthenelse{\boolean{longSol}}{\newpage}{}
% \section*{Aufgabe \theproblem \ifthenelse{\boolean{longSol}}{\ [.../\currfiledir]}{}}
%  \noindent
%  \begin{minipage}{\textwidth}
%    \begin{minipage}[t]{0.5\textwidth}
%      #2
%    \end{minipage}
%    \hfill
%    %\mbox{\numexpr{0.95-#5}} geht vielleicht auch irgendwie
%    \begin{minipage}[t]{0.5\textwidth}				%ehemals 0.45 [LARS]
%      ~\\[-1ex]%fakezeile, um beide minipages mit der t-Zeile auszurichten
%      #4 %argumentliste mit psfrag-befehlen
%      \centerline{\includegraphics[#1]{#3}}
%    \end{minipage}
%  \end{minipage}
%}
%
%
%% umgebung für Aufgaben mit zweispaltigem Bild
%\newcommand{\bigproblem}[4][width=0.8 \textwidth]{
% \addtocounter{problem}{1}
% 
% %insert page break if longsolution is provided
% \ifthenelse{\boolean{longSol}}{\newpage}{}
% \section*{Aufgabe \theproblem}
% \begin{minipage}{\textwidth}
%  \noindent
%      #2 \bigskip
%      
%  {
%      #4 %argumentliste mit psfrag-befehlen
%      \centerline{\includegraphics[#1]{#3}}
%  }
%  \end{minipage}
%}
%
%\newcommand{\solution}[2]
%{
%  \ifthenelse{\boolean{Solution}}
%  {
%  \ifthenelse{\NOT\boolean{longSol}}
%    {%short solution
%     \vspace*{-2ex}
%     \paragraph{Lösung:} #1
%    }
%    {%long solution
%     \subsection*{Lösung:} #2
%    }
%  }
%  {%else; not used here
%  }
%}
%
